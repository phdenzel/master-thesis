% !TEX root = ../main.tex
% Appendix B

\chapter{Derivation of the radiative transfer equations} % Main appendix title

\label{AppendixB} % For referencing this appendix elsewhere, use \ref{AppendixB}

The radiative transfer equation \eqref{eq:Radiative_transfer} can easily be derived using a common form for the distribution function, applying it to photons, to describe the energy of an infinitesimal 6 dimensional phase--space volume element.
\begin{align*}
 \mathrm{d}N = f(\textbf{x}, \textbf{p}, t)\,\mathrm{d}^{3}x\,\mathrm{d}^{3}p
\end{align*}

We know that the momentum of photons $\textbf{p} = \frac{h\nu}{c}\hat{\textbf{n}}$ can be described by their frequency $\nu$ and the direction they travel $\hat{\textbf{n}}$, while their energy is given by $E = h\nu$.
During their travel, photons, bundled as light rays, cover a certain surface area $\mathrm{d}S$ which determines the spatial volume element through the traveled path $c\mathrm{d}t$, if their curvature can be approximated to be radial, that is if the trajectory of photons is not bent.
The individual volume elements of phase--space are thus
\begin{align*}
 \mathrm{d}^{3}x = \mathrm{d}S\,c\mathrm{d}t\,; \qquad\qquad \mathrm{d}^{3}p &= p^{2}\mathrm{d}p\,\mathrm{d}\Omega = \Big(\frac{h\nu}{c}\Big)^{2}\,\Big(\frac{h\mathrm{d}\nu}{c}\Big)\,\mathrm{d}\Omega\,; \qquad\qquad \mathrm{d}N = \frac{\mathrm{d}E}{h\nu}
\end{align*}
\vspace{-0.75cm}
\begin{align*}
 \Longrightarrow\qquad
 \underbrace{\mathrm{d}N}_{\mathclap{\frac{\mathrm{d}E}{h\nu}}} &= f\,\overbrace{\mathrm{d}^{3}x}^{\mathclap{\mathrm{d}S\,c\mathrm{d}t}}\,\underbrace{\mathrm{d}^{3}p}_{\mathclap{(\frac{h\nu}{c})^{2}\,(\frac{h\mathrm{d}\nu}{c})\,\mathrm{d}\Omega}} \\
 \Longleftrightarrow\qquad
 \mathrm{d}E &= h\nu\,f\,\mathrm{d}S\,c\mathrm{d}t\,\Big(\frac{h\nu}{c}\Big)^{2}\,\Big(\frac{h\mathrm{d}\nu}{c}\Big)\,\mathrm{d}\Omega = \underbrace{\frac{h^{4}\nu^{3}}{c^{2}}f}_{\equiv I_{\nu}}\,\mathrm{d}S\,\mathrm{d}t\,\mathrm{d}\nu\,\mathrm{d}\Omega = I_{\nu}\,\mathrm{d}S\,\mathrm{d}t\,\mathrm{d}\nu\,\mathrm{d}\Omega
\end{align*}
where we used \eqnref{eq:dE_intensity} to identify the specific radiation intensity $I_{\nu}$.
Since the distribution function and specific radiation intensity only scale differently with $\nu$, we can interchange them in the following considerations.
Applying this relation to the Boltzmann equation \eqref{eq:BoltzmannEQ}, and ignoring the collision term for now, gives the radiative transfer equation in vacuum.
\begin{align*}
 &\frac{\partial f}{\partial t} + \overbrace{\textbf{v} }^{\mathclap{c\cdot\hat{\textbf{n}}}}\cdot\nabla f + \underbrace{ \dot{\textbf{v}} }_{\mathclap{\text{no acceleration for photons}}} \frac{\partial f}{\partial \textbf{v}} = 0 \\
 \Longleftrightarrow\qquad
 &\frac{\partial f}{\partial t} + c\cdot\hat{\textbf{n}}\cdot\nabla f = 0 \\
 \Longleftrightarrow\qquad
 &\frac{\partial I_{\nu}}{\partial t} + c\cdot\hat{\textbf{n}}\cdot\nabla I_{\nu} = 0
\end{align*}
Since there is a strong similarity to the Euler equations from kinetic theory, the equation above is sometimes also written in a Lagrangian-like form with the derivative $\frac{\mathrm{d}}{\mathrm{d}s} \equiv \frac{1}{c}\frac{\partial}{\partial t} + \hat{\textbf{n}}\cdot\nabla$.
\begin{align*}
 \Longleftrightarrow\qquad
 &\frac{1}{c}\frac{\partial I_{\nu}}{\partial t} + \hat{\textbf{n}}\cdot\nabla I_{\nu} = 0 \\
 \Longleftrightarrow\qquad
 &\frac{dI_{\nu}}{ds} = 0
\end{align*}

As explained in \secref{sec:RadiativeTransfer} with \eqnref{eq:Source_term_derivation}, the radiative transfer equation \eqref{eq:Radiative_transfer} for a non-vacuum regime is given by the previous equation, except for a non-zero source integral.

Using the same exemplary case, we yield

\begin{align*}
 \Aboxed{\frac{1}{c}\frac{\partial I_{\nu}}{\partial t} + \hat{\textbf{n}}\cdot\nabla I_{\nu} = j_{\nu} - \alpha_{\nu}I_{\nu}}
\end{align*}

This derivation is not without its assumptions and postulates of course.
For instance the assumption that the photon transport can be described by the Boltzmann equation was postulated or that scattering can be averaged over all directions equally.
An alternative derivation is to use the Maxwell equations to obtain the transport equation, if the micro--physical properties of arbitrarily formed, scattering particles are included in the calculations.

The same way as we did for kinetic theory in Appendix~\ref{AppendixA}, we can now take the moments of the radiative transfer equation using the definitions from Eqs.~\eqref{eq:spec_energy}, \eqref{eq:spec_flux} and \eqref{eq:spec_press}, but this time angular moments instead of collisional invariant.
\begin{itemize}
 \item The first moment ($\int_{4\pi} \mathrm{d}\Omega$) of the radiative transfer equation reads
  \begin{align*}
   &\int \frac{1}{c}\frac{\partial I_{\nu}}{\partial t} \,\mathrm{d}\Omega \,+\, \int \hat{\textbf{n}}\cdot\nabla I_{\nu} \,\mathrm{d}\Omega = \int j_{\nu} \,\mathrm{d}\Omega \,-\, \int \alpha_{\nu}I_{\nu} \,\mathrm{d}\Omega \\
   \Longleftrightarrow\qquad
   &\frac{\partial}{\partial t}\Big(\underbrace{ \int \frac{I_{\nu}}{c} \,\mathrm{d}\Omega }_{E_{\nu}}\Big) \,+\, \nabla\cdot\underbrace{ \int \hat{\textbf{n}}I_{\nu} \,\mathrm{d}\Omega }_{\textbf{F}_{\nu}} = j_{\nu}\underbrace{ \int\!\mathrm{d}\Omega }_{4\pi} \,-\, \alpha_{\nu}\underbrace{ \int I_{\nu} \,\mathrm{d}\Omega }_{cE_{\nu}} \\
   \Longleftrightarrow\qquad\!\!
   \Aboxed{&\frac{\partial E_{\nu}}{\partial t} \,+\, \nabla\cdot\textbf{F}_{\nu} = 4\pi j_{\nu}\,-\, \alpha_{\nu}cE_{\nu}}
  \end{align*}
  As always, we used the Leibniz integral rule in the first step, which finally yields the radiation energy conservation equation.

  \item The second moment ($\int_{4\pi}\hat{\textbf{n}}\,\mathrm{d}\Omega$) of the radiative transfer equation reads
  \begin{align*}
   &\int \hat{\textbf{n}}\frac{1}{c}\frac{\partial I_{\nu}}{\partial t} \,\mathrm{d}\Omega \,+\, \int (\hat{\textbf{n}}\otimes\hat{\textbf{n}})\cdot\nabla I_{\nu} \,\mathrm{d}\Omega = \int \hat{\textbf{n}} j_{\nu} \,\mathrm{d}\Omega \,-\, \int \hat{\textbf{n}} \alpha_{\nu}I_{\nu} \,\mathrm{d}\Omega \\
   \Longleftrightarrow\qquad
   &\frac{1}{c}\frac{\partial}{\partial t}\Big(\underbrace{ \int \hat{\textbf{n}}I_{\nu} \,\mathrm{d}\Omega }_{\textbf{F}_{\nu}}\Big) \,+\, \nabla\cdot\underbrace{ \int (\hat{\textbf{n}}\otimes\hat{\textbf{n}}) I_{\nu} \,\mathrm{d}\Omega }_{c\mathbb{P}_{\nu}} = \underbrace{ \int \hat{\textbf{n}} j_{\nu} \,\mathrm{d}\Omega }_{=0 \,(\text{isotropic})} \,-\, \alpha_{\nu}\underbrace{ \int \hat{\textbf{n}}I_{\nu} \,\mathrm{d}\Omega }_{\textbf{F}_{\nu}} \\
   \Longleftrightarrow\qquad
   &\frac{1}{c}\frac{\partial\textbf{F}_{\nu}}{\partial t} \,+\, c\,\nabla\cdot\mathbb{P}_{\nu} = - \alpha_{\nu}\textbf{F}_{\nu} \\
   \Longleftrightarrow\qquad\!\!
   \Aboxed{&\frac{1}{c^{2}}\frac{\partial\textbf{F}_{\nu}}{\partial t} \,+\, \nabla\cdot\mathbb{P}_{\nu} = - \frac{\alpha_{\nu}\textbf{F}_{\nu}}{c}}
  \end{align*}
  The vanishing emission term can also be explained mathematically, if one considers
  \begin{align*}
   \int_{4\pi}\hat{\textbf{n}} j_{\nu}\,\mathrm{d}\Omega &= j_{\nu} \int_{0}^{2\pi}\int_{0}^{\pi}\big(\sin{\theta}\cos{\phi}\,\hat{\textbf{x}} + \sin{\theta}\sin{\phi}\,\hat{\textbf{y}} + \cos{\theta}\,\hat{\textbf{z}} \big)\,\mathrm{d}\theta\sin{\theta}\mathrm{d}\phi \\
   &= j_{\nu} \int_{0}^{2\pi}\int_{0}^{\pi}\big(\underbrace{ \sin^{2}{\theta} }_{\mathclap{\frac{1}{2} - \frac{\cos{2\theta}}{2}}}\cos{\phi}\,\hat{\textbf{x}} + \sin^{2}{\theta}\sin{\phi}\,\hat{\textbf{y}} + \sin{\theta}\cos{\theta}\,\hat{\textbf{z}} \big)\,\mathrm{d}\theta\mathrm{d}\phi \\
   &= j_{\nu} \underbrace{ [\sin{\phi}]_{0}^{2\pi} }_{0}\big[\frac{\theta}{2}-\frac{\sin{\theta}\cos{\theta}}{2}\big]_{0}^{\pi}\,\hat{\textbf{x}} + j_{\nu} \underbrace{ [-\cos{\phi}]_{0}^{2\pi} }_{0}\big[\frac{\theta}{2} - \frac{\sin{\theta}\cos{\theta}}{2}\big]_{0}^{\pi}\,\hat{\textbf{y}} \\
   &\quad\! + 2\pi\underbrace{ \big[-\frac{\cos^{2}{\theta}}{2}\big]_{0}^{\pi} }_{0}\,\hat{\textbf{z}} \\
   &= 0
  \end{align*}

\end{itemize}
