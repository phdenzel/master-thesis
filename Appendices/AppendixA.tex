% !TEX root = ../main.tex
% Appendix A

\chapter{Derivation of the Euler equations} % Main appendix title

\label{AppendixA} % For referencing this appendix elsewhere, use \ref{AppendixA}

The Euler equations Eqs.~\eqref{eq:EulerMass}, \eqref{eq:EulerMomentum}, and \eqref{eq:EulerEnergy} are derived by taking moments, i.e., $m\cdot\int_{\mathbb{R}^{3}} \mathrm{d}^{3}v$, $m\cdot\int_{\mathbb{R}^{3}}\,\textbf{v}\,\mathrm{d}^{3}v$ and $m\cdot\int_{\mathbb{R}^{3}}\,\frac{v^{2}}{2}\,\mathrm{d}^{3}v$, of the Boltzmann equation \eqref{eq:BoltzmannEQ}.

\begin{equation*}
  \frac{\partial f}{\partial t} + \dot{\textbf{x}} \frac{\partial f}{\partial \textbf{x}} + \dot{\textbf{v}} \frac{\partial f}{\partial \textbf{v}} = \Big(\frac{Df}{Dt}\Big)_{\text{coll.}}
\end{equation*}

Since these moments represent collisional invariants, the collision integral is averaged to zero (see \secref{sec:HydrodynamicEquations}).
\begin{itemize}
 \item The first moment ($m\cdot\int \mathrm{d}^{3}v$) of the Boltzmann equation reads
 \begin{align*}
  &0 = m \int\,\frac{\partial f}{\partial t}\,\mathrm{d}^{3}v \,+\, m \int\,\dot{\textbf{x}}\,\frac{\partial f}{\partial \textbf{x}}\,\mathrm{d}^{3}v \,+\, m \int\,\dot{\textbf{v}}\,\frac{\partial f}{\partial \textbf{v}}\,\mathrm{d}^{3}v \\
  \Longleftrightarrow\qquad
  &0 = \frac{\partial}{\partial t} \underbrace{ \int\,m  f\,\mathrm{d}^{3}v }_{\rho(\textbf{x}, t)} \,+\, \underbrace{ \frac{\partial}{\partial \textbf{x}} }_{\nabla} \int\,m \underbrace{ \dot{\textbf{x}} }_{\textbf{v}} f\,\mathrm{d}^{3}v \,+\, m \int\,\underbrace{ \dot{\textbf{v}} }_{\textbf{a}} \frac{\partial f}{\partial \textbf{v}}\,\mathrm{d}^{3}v \\
  \Longleftrightarrow\qquad
  &0 = \frac{\partial}{\partial t} \rho(\textbf{x}, t) \,+\, \nabla \underbrace{ \int\,m \textbf{v} f\,\mathrm{d}^{3}v }_{\rho(\textbf{x}, t)  \textbf{u}} \enskip+\enskip m\,\textbf{a} \underbrace{ \int\,\frac{\partial f}{\partial \textbf{v}}\,\mathrm{d}^{3}v }_{ \mathclap{[f(\infty) - f(-\infty)] \rightarrow 0} }
 \end{align*}
 In the first step, we have used the Leibniz integral rule, whereas in the second step we have used the moments of the distribution function itself, see Eqs.~\eqref{eq:fmoment1} and \eqref{eq:fmoment2}.
 Since $f$ is a well--behaved distribution function, it has the property: $f \rightarrow 0$ for $\textbf{v} \rightarrow \pm \infty$, from which follows $[f(\textbf{v} = \infty) - f(\textbf{v} = -\infty)] \rightarrow 0$.
 This yields the first Euler equation \eqref{eq:EulerMass}, representing the conservation of mass.
 \begin{align*}
  \Aboxed{\frac{\partial\rho}{\partial t} \,+\, \nabla(\rho\textbf{u}) = 0}
 \end{align*}

 \item The second moment ($m \int\,\textbf{v}\,\mathrm{d}^{3}v$) of the Boltzmann equation reads
 \begin{align*}
  &0 = m \int\,\textbf{v}\frac{\partial f}{\partial t}\,\mathrm{d}^{3}v \,+\, m \int\,\textbf{v}\,\dot{\textbf{x}}\,\frac{\partial f}{\partial \textbf{x}}\,\mathrm{d}^{3}v \,+\, m \int\,\textbf{v}\,\dot{\textbf{v}}\,\frac{\partial f}{\partial \textbf{v}}\,\mathrm{d}^{3}v \\
  \Longleftrightarrow\qquad
  &0 = \frac{\partial}{\partial t} \underbrace{ \int\,m \textbf{v}f\,\mathrm{d}^{3}v }_{\rho(\textbf{x}, t)\textbf{u}} \,+\, \underbrace{ \frac{\partial}{\partial \textbf{x}} }_{\nabla} \int\,m \underbrace{ \textbf{v}\,\dot{\textbf{x}} }_{\textbf{v}\otimes\textbf{v}} f\,\mathrm{d}^{3}v \,+\, m \int\,\textbf{v}\,\underbrace{ \dot{\textbf{v}} }_{\textbf{a}} \frac{\partial f}{\partial \textbf{v}}\,\mathrm{d}^{3}v \\
  \Longleftrightarrow\qquad
  &0 = \frac{\partial(\rho \textbf{u})}{\partial t} \,+\, \nabla \int\,m \big(\textbf{v}\otimes\textbf{v}\big) f\,\mathrm{d}^{3}v \,+\, m \textbf{a}\,\underbrace{ \int\,\textbf{v}\,\frac{\partial f}{\partial \textbf{v}}\,\mathrm{d}^{3}v }_{\mathclap{\int\,\alpha\beta'\,\mathrm{d}s = \alpha\beta\mid_{-\infty}^{\infty} - \int\,\alpha'\beta\,\mathrm{d}s}}
 \end{align*}
 Again, we have used the Leibniz integral rule and the definitions of the distribution function's moments.
 Here, the second term can be written with $\textbf{v} = \textbf{u} + \textbf{w}$, where $\textbf{w}$ describes the thermal velocity with an average of $\langle \textbf{w}\rangle = 0$, such that the average fluid velocity still remains $\langle \textbf{v}\rangle = \textbf{u}$.
 Using integration by parts on the third term with $\alpha = \textbf{v}$, $\alpha' = 1$, $\beta' = \frac{\partial f}{\partial\textbf{v}}$, and $\beta = f$, we get
 \begin{align*}
 \Longleftrightarrow\qquad
  &0 = \frac{\partial(\rho \textbf{u})}{\partial t} \,+\, \nabla \int\,m \big(\textbf{v}\otimes\textbf{v}\big) f\,\mathrm{d}^{3}v \,+\, m \textbf{a}\,\big(\underbrace{ \textbf{v}\,f\mid_{-\infty}^{\infty} }_{\mathclap{f \rightarrow 0 \text{ for } \textbf{v} \rightarrow \pm \infty}} - \underbrace{ \int\,f\mathrm{d}^{3}v }_{n(\textbf{x}, t)}\big)
 \end{align*}
 where $n(\textbf{x}, t) = \frac{\rho(\textbf{x}, t)}{m}$ is the number density.
 Again, the distribution function is well--behaved, which is why the first term in the parenthesis goes to zero.
 From this point on, the second term is easier to understand if we make use of Einstein's notation
 \begin{align*}
 \Longleftrightarrow\qquad
  &0 = \frac{\partial(\rho u_{i})}{\partial t} \,+\, \nabla \int\,m \,\underbrace{ \big(v_{i}v_{j}\big) }_{\mathclap{u_{i}u_{j} + w_{i}w_{j} + u_{i}w_{j} + u_{j}w_{i}}} f\,\mathrm{d}^{3}v \,-\, \rho\,a_{i} \\
  \Longleftrightarrow\qquad
  &0 = \frac{\partial(\rho u_{i})}{\partial t} \,+\, \nabla \int\,m \,\big(u_{i}u_{j} + w_{i}w_{j} + u_{i}w_{j} + u_{j}w_{i}\big) f\,\mathrm{d}^{3}v \,-\, \rho\,a_{i} \\
  \Longleftrightarrow\qquad
  &0 = \frac{\partial(\rho u_{i})}{\partial t} \,+\, \nabla m \,\big( \int u_{i}u_{j} f\,\mathrm{d}^{3}v + \int w_{i}w_{j} f\,\mathrm{d}^{3}v \\
   &\hspace{76pt} + \int u_{i}w_{j} f\,\mathrm{d}^{3}v + \int u_{j}w_{i} f\,\mathrm{d}^{3}v\big) \,-\, \rho\,a_{i} \\
  \Longleftrightarrow\qquad
  &0 = \frac{\partial(\rho u_{i})}{\partial t} \,+\, \nabla m \,\big( \int u_{i}u_{j} f\,\mathrm{d}^{3}v + \int w_{i}w_{j} f\,\mathrm{d}^{3}v \\
   &\hspace{66pt} + u_{i} \underbrace{ \int w_{j} f\,\underbrace{ \mathrm{d}^{3}v }_{\mathrm{d}^{3}w} }_{n(\textbf{x}, t)\langle \textbf{w}\rangle = 0} \,+\, u_{j} \underbrace{ \int w_{i} f\,\underbrace{ \mathrm{d}^{3}v }_{\mathrm{d}^{3}w} }_{n(\textbf{x}, t)\langle \textbf{w}\rangle = 0}\big) \,-\, \rho\,a_{i} \\
  \Longleftrightarrow\qquad
  &0 = \frac{\partial(\rho u_{i})}{\partial t} \,+\, \nabla m \,\big( \underbrace{ \int u_{i}u_{j} f\,\mathrm{d}^{3}v }_{(u_{i}u_{j})n(\textbf{x}, t)} + \int w_{i}w_{j} f\,\mathrm{d}^{3}v\big) \,-\, \rho\,a_{i} \\
  \Longleftrightarrow\qquad
  &0 = \frac{\partial(\rho u_{i})}{\partial t} \,+\, \nabla \big( \rho(u_{i}u_{j}) + \underbrace{ \int m  w_{i}w_{j} f\,\mathrm{d}^{3}v}_{\equiv \mathbb{P}_{ij}} \big) \,-\, \rho\,a_{i}
 \end{align*}
 In the last step, we used the definition for the pressure tensor $\mathbb{P}_{ij} \equiv \int m  w_{i}w_{j} f\,\mathrm{d}^{3}w$.
 Rewriting this final equation again in vectorial form, recovers the second Euler equation \eqref{eq:EulerMomentum}.
 \begin{align*}
  \Aboxed{\frac{\partial(\rho \textbf{u})}{\partial t} \,+\, \nabla \big( \rho(\textbf{u}\otimes\textbf{u}) + \mathbb{P} \big) = \rho\,\textbf{a}}
 \end{align*}

 \item The third moment ($m \int_{\mathbb{R}^{3}}\,\frac{v^{2}}{2}\,\mathrm{d}^{3}v$) of the Boltzmann equation reads
 \begin{align*}
  &0 = m \int\,\underbrace{ \frac{v^{2}}{2} }_{\mathclap{\frac{u^{2}}{2}+\frac{w^{2}}{2}+\textbf{u}\cdot\textbf{w}}}\frac{\partial f}{\partial t}\,\mathrm{d}^{3}v \,+\, m \int\,\underbrace{ \frac{v^{2}}{2} }_{\mathclap{\frac{u^{2}}{2}+\frac{w^{2}}{2}+\textbf{u}\cdot\textbf{w}}}\,\underbrace{ \dot{\textbf{x}} }_{\textbf{v}}\,\frac{\partial f}{\partial \textbf{x}}\,\mathrm{d}^{3}v \,+\, m \int\,\frac{v^{2}}{2}\,\underbrace{ \dot{\textbf{v}} }_{\textbf{a}}\,\frac{\partial f}{\partial \textbf{v}}\,\mathrm{d}^{3}v \\
  \Longleftrightarrow\qquad
  &0 = \frac{\partial}{\partial t}\Big( \int m \frac{u^{2}}{2}f\,\mathrm{d}^{3}v + \underbrace{ \int\,m \frac{w^{2}}{2}f\,\underbrace{ \mathrm{d}^{3}v }_{\mathrm{d}^{3}w} }_{\mathclap{\equiv\,\rho\epsilon}} + m\,\textbf{u}\underbrace{ \int m\,\textbf{w}\,f\,\underbrace{ \mathrm{d}^{3}v }_{\mathrm{d}^{3}w} }_{n(\textbf{x}, t)\langle \textbf{w}\rangle = 0} \Big) \\
   &\hspace{3pt} +\, \underbrace{ \frac{\partial}{\partial\textbf{x}} }_{\nabla}\Big( \int m\,\frac{u^{2}}{2}\textbf{v}\,f\,\mathrm{d}^{3}v + \int\,m\,\frac{w^{2}}{2}\textbf{v}\,f\,\mathrm{d}^{3}v + \int\,m\,\big(\textbf{u}\cdot\textbf{w}\big)\textbf{v} f\,\mathrm{d}^{3}v \Big) \\
   &\hspace{3pt} +\, m\,\textbf{a} \underbrace{\int\,\frac{v^{2}}{2}\,\frac{\partial f}{\partial \textbf{v}}\,\mathrm{d}^{3}v }_{\mathclap{\int\,\alpha\beta'\,\mathrm{d}s = \alpha\beta\mid_{-\infty}^{\infty} - \int\,\alpha'\beta\,\mathrm{d}s}}
 \end{align*}
 As in the derivations before, we have made use of the Leibniz integral rule, the thermal velocity $\textbf{w} = \textbf{v} - \textbf{u}$ with $\langle \textbf{w}\rangle = 0$, and the definitions of the distribution function's moments.
 In the third term, we --- again --- integrate by parts, with $\alpha = \frac{v^{2}}{2}$, $\alpha' = \textbf{v}$, $\beta' = \frac{\partial f}{\partial\textbf{v}}$, and $\beta = f$, and subsequently remember that $f \rightarrow 0$ for $\textbf{v} \rightarrow \pm \infty$, since $f$ is a distribution.
 \begin{align*}
  \Longleftrightarrow\qquad
  &0 = \frac{\partial}{\partial t}\Big( \frac{u^{2}}{2}\underbrace{ \int m\,f\,\mathrm{d}^{3}v }_{\rho(\textbf{x}, t)} + \rho\epsilon \Big) \\
   &\hspace{3pt} +\, \nabla\Big( \frac{u^{2}}{2}\underbrace{ \int m\,\textbf{v}\,f\,\mathrm{d}^{3}v }_{\rho(\textbf{x}, t)\textbf{u}} + \int\,m\,\underbrace{ \textbf{v} }_{\mathclap{\textbf{u}+\textbf{w}}} \frac{w^{2}}{2}f\,\mathrm{d}^{3}v + \int\,m\,\big(\textbf{u}\cdot\textbf{w}\big)\underbrace{ \textbf{v} }_{\mathclap{\textbf{u}+\textbf{w}}} f\,\mathrm{d}^{3}v \Big) \\
   &\hspace{3pt} +\, m\,\textbf{a} \big( \underbrace{ \frac{v^{2}}{2}\,f\mid_{-\infty}^{\infty} }_{\mathclap{f \rightarrow 0 \text{ for } \textbf{v} \rightarrow \pm \infty}} - \underbrace{ \int\textbf{v}\,f\,\mathrm{d}^{3}v }_{n(\textbf{x}, t)\textbf{u}} \big) \\
  \Longleftrightarrow\qquad
  &0 = \frac{\partial}{\partial t}\Big( \frac{1}{2}\rho\,u^{2} + \rho\epsilon \Big) \\
   &\hspace{3pt} +\, \nabla\Big( \frac{1}{2}\rho\,u^{2}\,\textbf{u} + \underbrace{ \int\,m\,\big(\textbf{u}+\textbf{w}\big) \frac{w^{2}}{2}f\,\mathrm{d}^{3}v }_{I_{1}} + \underbrace{ \int\,m\,\big(\textbf{u}\cdot\textbf{w}\big)\big(\textbf{u}+\textbf{w}\big) f\,\mathrm{d}^{3}v }_{I_{2}} \Big) \\
   &\hspace{3pt} -\, \rho\,\textbf{a}\,\textbf{u}
 \end{align*}
 For sake of simplicity, we now look at the integrals $I_{1}$ and $I_{2}$ separately. And again, for $I_{2}$ we use Einstein's notation
 \begin{align*}
  I_{1} &= \int m\,\big(\textbf{u}+\textbf{w}\big)\frac{w^{2}}{2}\,f\,\mathrm{d}^{3}v \\
	&= \int m\,\textbf{u}\,\frac{w^{2}}{2}\,f\,\mathrm{d}^{3}v \,+\, \int m\,\textbf{w}\,\frac{w^{2}}{2}\,f\,\mathrm{d}^{3}v \\
	&= \textbf{u}\underbrace{ \int m\,\frac{w^{2}}{2}\,f\,\underbrace{ \mathrm{d}^{3}v }_{\mathclap{\mathrm{d}^{3}w}} }_{\mathclap{\equiv\,\rho\epsilon}} \,+\, \underbrace{ \int m\,\textbf{w}\,\frac{w^{2}}{2}\,f\,\mathrm{d}^{3}v }_{\equiv \textbf{Q}} \\
	&= \rho\epsilon\cdot\textbf{u} + \textbf{Q} \\
  \\
  I_{2} &= \int\,m\,\big(\textbf{u}\cdot\textbf{w}\big)\big(\textbf{u}+\textbf{w}\big) f\,\mathrm{d}^{3}v \\
	&= \int\,m\,\big(\textbf{u}\cdot\textbf{w}\big)\,\textbf{u}\,f\,\mathrm{d}^{3}v \,+\, \int\,m\,\big(\textbf{u}\cdot\textbf{w}\big)\,\textbf{w}\,f\,\mathrm{d}^{3}v \\
	&= \int\,m\,u_{i}\,w_{i}\,u_{j}\,f\,\mathrm{d}^{3}v \,+\, \int\,m\,u_{i}\,w_{i}\,w_{j}\,f\,\mathrm{d}^{3}v \\
	&= u_{i}\,u_{j}\underbrace{ \int\,m\,w_{i}\,f\,\mathrm{d}^{3}v }_{n(\textbf{x}, t)\langle \textbf{w}\rangle = 0} \,+\, \underbrace{ u_{i}\int\,m\,w_{i}\,w_{j}\,f\,\mathrm{d}^{3}v }_{\equiv \mathbb{P}_{ij}\,u_{i}} \\
	&= \mathbb{P}_{ij}\,u_{i} \,=\, \mathbb{P}\,\textbf{u}
 \end{align*}
 Re--substituting both expressions for the integrals $I_{1}$ and $I_{2}$ into the equation before, yields
 \begin{align*}
  &0 = \frac{\partial}{\partial t}\underbrace{ \Big( \frac{1}{2}\rho\,u^{2} + \rho\epsilon \Big) }_{\equiv E} \,+\, \nabla\Big( \underbrace{ \big(\frac{1}{2}\rho\,u^{2} + \rho\epsilon\big) }_{\equiv E} \textbf{u} + \mathbb{P}\,\textbf{u} + \textbf{Q} \Big) \,-\, \rho\,\textbf{a}\,\textbf{u}
 \end{align*}
 The energy density is defined as $E = \frac{1}{2}\rho\,u^{2} + \rho\,\epsilon$ and is the conserved quantity in the third Euler equation \eqref{eq:EulerEnergy} which is the equation above, if one neglects the heat flux term $\textbf{Q}$.
 \begin{align*}
  \Aboxed{\frac{\partial E}{\partial t} \,+\, \nabla\Big(E + \mathbb{P}\Big)\,\textbf{u} = \rho\,\textbf{a}\,\textbf{u}}
 \end{align*}

 \item From this point, knowing that the Lagrangian derivative $\frac{D}{Dt} \equiv \frac{\partial}{\partial t} + \textbf{u} \cdot \nabla$ is also linear, going from Eulerian to Lagrangian form of the equations is not hard.

 \eqref{eq:EulerMass} to \eqref{eq:EulerMassL}:
 \vspace{-0.5cm}
 \begin{align*}
  &\frac{\partial\rho}{\partial t} \,+\, \underbrace{ \nabla(\rho\textbf{u}) }_{\mathclap{\rho\,\nabla\cdot\textbf{u} + \textbf{u}\cdot\nabla\,\rho}} = 0 \\
  \Longleftrightarrow\qquad
  &\underbrace{ \frac{\partial\rho}{\partial t} \,+\, \textbf{u}\cdot\nabla\,\rho }_{\frac{D\rho}{Dt}} \,+\, \rho\,\nabla\cdot\textbf{u} = 0 \\
  \Longleftrightarrow\qquad\!\!
  \Aboxed{&\frac{1}{\rho}\frac{D\rho}{Dt} = -\nabla\cdot\textbf{u}}
 \end{align*}
 \eqref{eq:EulerMomentum} to \eqref{eq:EulerMomentumL}:
 \vspace{-0.5cm}
 \begin{align*}
  &\frac{\partial(\rho \textbf{u})}{\partial t} \,+\, \nabla \big( \rho(\textbf{u}\otimes\textbf{u}) + \mathbb{P} \big) = \rho\,\textbf{a} \\
  \Longleftrightarrow\qquad
  &\textbf{u}\frac{\partial\rho}{\partial t} + \rho\frac{\partial\textbf{u}}{\partial t} \,+\, \underbrace{ \nabla\big(\rho(\textbf{u}\otimes\textbf{u})\big) }_{\mathclap{\rho\big(\textbf{u}(\nabla\cdot\textbf{u}) + (\textbf{u}\cdot\nabla)\textbf{u}\big) + \textbf{u}\big(\textbf{u}\cdot\nabla\,\rho\big)}} = -\nabla\mathbb{P} + \rho\,\textbf{a} \\
  \Longleftrightarrow\qquad
  &\underbrace{ \textbf{u}\frac{\partial\rho}{\partial t} + \textbf{u}\big(\textbf{u}\cdot\nabla\,\rho\big) }_{\mathclap{\textbf{u}\frac{D\rho}{Dt} = \,-\rho\textbf{u}\big(\nabla\cdot\textbf{u}\big)}} \,+\, \underbrace{ \rho\frac{\partial\textbf{u}}{\partial t} + \rho(\textbf{u}\cdot\nabla)\textbf{u} }_{\mathclap{\rho\frac{D\textbf{u}}{Dt}}} \,+\, \rho\textbf{u}(\nabla\cdot\textbf{u}) = -\nabla\mathbb{P} + \rho\,\textbf{a} \\
  \Longleftrightarrow\qquad
  &\rho\frac{D\textbf{u}}{Dt} \,-\, \rho\textbf{u}\big(\nabla\cdot\textbf{u}\big) \,+\, \rho\textbf{u}(\nabla\cdot\textbf{u}) = -\nabla\mathbb{P} + \rho\,\textbf{a} \\
  \Longleftrightarrow\qquad\!\!
  \Aboxed{&\rho\frac{D\textbf{u}}{Dt} = -\nabla\mathbb{P} + \rho\,\textbf{a}}
 \end{align*}
 \eqref{eq:EulerEnergy} to \eqref{eq:EulerEnergyL}:
 \begin{align*}
  &\frac{\partial E}{\partial t} \,+\, \underbrace{ \nabla\big(E+\mathbb{P}\big)\textbf{u} }_{\mathclap{\textbf{u}\cdot\nabla E + E(\nabla\cdot\textbf{u}) + \textbf{u}\cdot\nabla\mathbb{P} + \mathbb{P}\,\nabla\cdot\textbf{u}}} = \rho\textbf{a}\textbf{u} \\
  \Longleftrightarrow\qquad
  &\underbrace{ \frac{\partial E}{\partial t} + \textbf{u}\cdot\nabla E }_{\mathclap{\frac{DE}{Dt}}} + E(\underbrace{ \nabla\cdot\textbf{u} }_{\mathclap{-\frac{1}{\rho} \frac{D\rho}{Dt}}}) + \textbf{u}\cdot\nabla\mathbb{P} + \mathbb{P}\,\nabla\cdot\textbf{u} = \rho\textbf{a}\textbf{u} \\
  \Longleftrightarrow\qquad
  &\underbrace{ \frac{DE}{Dt} }_{\mathclap{\hspace{75pt}\frac{D}{Dt}\big(\frac{1}{2}\rho u^{2} + \rho\epsilon\big) = \frac{1}{2}u^{2}\frac{D\rho}{Dt} + \rho\textbf{u}\frac{D\textbf{u}}{Dt} + \epsilon\frac{D\rho}{Dt} + \rho\frac{D\epsilon}{Dt} }} - \frac{E}{\rho} \frac{D\rho}{Dt} + \textbf{u}\cdot\nabla\mathbb{P} + \mathbb{P}\,\nabla\cdot\textbf{u} = \rho\textbf{a}\textbf{u} \\
  \Longleftrightarrow\qquad
  &\frac{1}{2}u^{2}\frac{D\rho}{Dt} + \underbrace{ \rho\textbf{u}\frac{D\textbf{u}}{Dt} }_{\mathclap{\textbf{u}\big(-\nabla\mathbb{P} + \rho\textbf{a}\big)}} + \epsilon\frac{D\rho}{Dt} + \rho\frac{D\epsilon}{Dt} - \underbrace{ \frac{E}{\rho} \frac{D\rho}{Dt} }_{\mathclap{\frac{1}{2}u^{2}\frac{D\rho}{Dt} + \epsilon\frac{D\rho}{Dt}}} + \textbf{u}\cdot\nabla\mathbb{P} + \mathbb{P}\,\nabla\cdot\textbf{u} = \rho\textbf{a}\textbf{u} \\
  \Longleftrightarrow\qquad
  &\frac{1}{2}u^{2}\frac{D\rho}{Dt} - \frac{1}{2}u^{2}\frac{D\rho}{Dt} + \epsilon\frac{D\rho}{Dt} - \epsilon\frac{D\rho}{Dt} + \textbf{u}\cdot\nabla\mathbb{P} - \textbf{u}\cdot\nabla\mathbb{P} + \rho\frac{D\epsilon}{Dt} + \mathbb{P}\,\nabla\cdot\textbf{u} = \rho\textbf{a}\textbf{u} - \rho\textbf{a}\textbf{u} \\
  \Longleftrightarrow\qquad\!\!
  \Aboxed{&\rho\frac{D\epsilon}{Dt} = - \mathbb{P}\,\nabla\cdot\textbf{u}}
 \end{align*}
 For \eqnref{eq:EulerMomentumL} the result before was used, i.e., \eqnref{eq:EulerMassL}, and in a similar manner for \eqnref{eq:EulerEnergyL} both results before were used, i.e. Eqs.~\eqref{eq:EulerMassL} and \eqref{eq:EulerMomentumL}.

\end{itemize}
